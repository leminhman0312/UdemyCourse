% Created 2021-08-23 Mon 19:45
% Intended LaTeX compiler: pdflatex
\documentclass[11pt]{article}
\usepackage[utf8]{inputenc}
\usepackage[T1]{fontenc}
\usepackage{graphicx}
\usepackage{grffile}
\usepackage{longtable}
\usepackage{wrapfig}
\usepackage{rotating}
\usepackage[normalem]{ulem}
\usepackage{amsmath}
\usepackage{textcomp}
\usepackage{amssymb}
\usepackage{capt-of}
\usepackage{hyperref}
\usepackage{minted}
\author{Max Le}
\usepackage{hyperref}
\usepackage{minted}
\date{Monday 23-08-2021}
\title{Linkedlist Operations (Singly and Doubly)}
\hypersetup{
 pdfauthor={},
 pdftitle={Linkedlist Operations (Singly and Doubly)},
 pdfkeywords={},
 pdfsubject={},
 pdfcreator={Emacs 27.2 (Org mode 9.4.4)}, 
 pdflang={English}}
\begin{document}

\maketitle
\tableofcontents

\section{Theory}
\label{sec:org2c4f776}
\subsection{Array vs Linkedlist}
\label{sec:org22100f3}
\subsubsection{Array}
\label{sec:org6e78577}
\begin{enumerate}
\item fixed size
\label{sec:org35ee98f}
\item create on stack e.g. int A[5]
\label{sec:org8db9558}
\item create on heap e.g. int *p = new int[5];
\label{sec:orgf35d1b9}
\end{enumerate}
\subsubsection{Linkedlist}
\label{sec:org3eb3ade}
\begin{enumerate}
\item Allocate memories in node:  pointer to current elemen | pointer to next element
\label{sec:org86c30cd}
\item Create on heap
\label{sec:org255dfd1}
\end{enumerate}
\subsection{LinkedList basic}
\label{sec:org16cc2ea}
\subsubsection{collections of nodes, each node has "Data" + "Pointer to next Node"}
\label{sec:org5bffec8}
\subsubsection{"Head" points to first node}
\label{sec:org2119e0b}
\subsubsection{Addresses may not be side by side, unlike array}
\label{sec:org676240a}
\subsubsection{To create a node, need: DATA and POINTER to a NODE (NODE type)}
\label{sec:orga64cc7c}
\subsubsection{A node structure:  DATA | NEXT}
\label{sec:org759fac9}
\subsubsection{For C language:}
\label{sec:orgaa80c08}
\begin{minted}[breaklines=true,breakanywhere=true,frame=single]{c}
struct Node
{
  int data;
  struct Node *next;		/* self referential structure */
};

int main(){
  struct Node *p;
  p = (struct Node*)malloc(sizeof(struct Node));
  /* For C++ */
  /* p = new Node */


  p->data = 10; 		/* assign 10 to DATA */
  p->next = 0;			/* or NULL, meaning next pointer points to nothing */

  return 0;
}
\end{minted}
\subsubsection{How to display a linkedlist}
\label{sec:org228e0a4}
\begin{enumerate}
\item Iterative
\label{sec:orgbf3fab7}
\begin{minted}[breaklines=true,breakanywhere=true,frame=single]{c}
struct Node *p = first;		/* set a pointer to point to first node */

while (p != 0) {
  printf("%d\n", p->data);	/* print the data */
  p = p ->next;			/* move p to next node */
 }
\end{minted}
\item Recursive
\label{sec:orgd9f2015}
\begin{minted}[breaklines=true,breakanywhere=true,frame=single]{c}
void displayRecursive(struct Node *p){
  if (p != NULL) {
    printf("%d\n", p->data);
    displayRecursive(p->next);
  }
}
\end{minted}
\end{enumerate}
\subsection{Singular Linked List}
\label{sec:org8bf64d3}
\subsubsection{How to count node in linked list}
\label{sec:orga1bb46f}
\begin{enumerate}
\item Iterative
\label{sec:org22c11e3}
\begin{minted}[breaklines=true,breakanywhere=true,frame=single]{c}
/* taking O(n) time and O(1) space */
int count(struct Node *p){
  int c = 0;
  while (p != NULL) {
    c++;
    p = p->next
      }
  return c;
}
\end{minted}
\item Recursive
\label{sec:orgb5d6338}
\begin{minted}[breaklines=true,breakanywhere=true,frame=single]{c}
/* Time O(n) Space O(n) */
int count (struct Node *p){
  if (p == NULL) {
    return 0;
  }
  else {
    return count(p->next) + 1;
  }
}
\end{minted}
\end{enumerate}
\subsubsection{How to sum all elements in linked list}
\label{sec:org5ac90c8}
\begin{enumerate}
\item Iterative
\label{sec:orgca801e5}
\begin{minted}[breaklines=true,breakanywhere=true,frame=single]{c}
/* Time O(n) Space O(1)*/
int Add(struct Node *p){
  int sum = 0;
  while (p != NULL) {
    sum = sum + p->data;
    p=p->next;
  }
  return sum;
}
\end{minted}
\item Recursion
\label{sec:org56329fe}
\begin{minted}[breaklines=true,breakanywhere=true,frame=single]{c}
/* Time, Space O(n) */
int Add(struct Node *p){
  if (p == NULL) {
    return 0;
  }
  else {
    return Add(p->next)+p->data;
  }
}
\end{minted}
\end{enumerate}
\subsubsection{Find max element in linked list}
\label{sec:orgc144fd7}
\begin{enumerate}
\item Iterative
\label{sec:orgeda1ce8}
\begin{minted}[breaklines=true,breakanywhere=true,frame=single]{c}
int max(struct Node *p){
  int m = -32768;		/* min integer */
  while (p != NULL) {
    if (p->data > m) {
      m = p->data;
    }
    p = p->next;
  }
  return m;
}
\end{minted}
\item Recursive
\label{sec:org25205d1}
\begin{minted}[breaklines=true,breakanywhere=true,frame=single]{c}
int max(struct Node *p){

  int x = 0;

  if (p == NULL) {
    return MIN_INT;
  }
  else {
    x = max(p->next);
    if (x > p->data) {
      return x;
    }
    else {
      return p->data;
    }
  }
}
\end{minted}

\begin{minted}[breaklines=true,breakanywhere=true,frame=single]{c}
int max(struct Node *p){
  int x = 0;
  if (p == 0) {
    return MIN_INT;
  }
  x = max(p->next);
  return x > p->data ? x : p->data;
}
\end{minted}
\end{enumerate}
\subsubsection{Searching (linear search)}
\label{sec:orgdce38ed}
\begin{enumerate}
\item Iterative
\label{sec:org698320f}
\begin{minted}[breaklines=true,breakanywhere=true,frame=single]{c}
Node *search (struct Node *p, int key){
  while (p != NULL) {
    if (key == p->data) {
      return(p);
    }
    p = p->next;
  }
  return NULL;
}
\end{minted}
\item Recursive
\label{sec:org27a09e6}
\begin{minted}[breaklines=true,breakanywhere=true,frame=single]{c}
Node *search(struct Node *p, int key){
  if (p == NULL) {
    return NULL;
  }

  if (key == p->data) {
    return p;
  }

  return search(p->next, key);
}
\end{minted}
\item Move found to head
\label{sec:orged35049}
\begin{minted}[breaklines=true,breakanywhere=true,frame=single]{c}
Node *search(struct Node *p, int key){
  Node *q = NULL; 		/* previous pointer */

  while (p != NULL) {
    if (key == p->data) {
      q->next = p->next;
      p->next = first;
      first = p;
    }
    q = p;
    p = p->next;
  }
}
\end{minted}
\end{enumerate}
\subsubsection{Inserting}
\label{sec:org7da9388}
\begin{enumerate}
\item Insert \textbf{BEFORE} first node
\label{sec:org1e3d18a}
\begin{minted}[breaklines=true,breakanywhere=true,frame=single]{c}
/* constant time */
Node *t = new Node; 		/* create new node */
t->data = x; 			/* assign new node data */
t->next = first; 		/* t points to first pointer, making t comes before first */
first = t; 			/* old "first" point to t , t is now first */
\end{minted}
\item Insert \textbf{AFTER} given position
\label{sec:orgef085c9}
\begin{minted}[breaklines=true,breakanywhere=true,frame=single]{c}
/* insert between left and right node */
/* O(N) max time, O(1) min time */
Node *t = new Node;
t->data = x;
p = first; 			/* start temporary pointer from first */
pos = 4; 			/* position to insert after */

/* moving p till reach left node */
for (i = 0; i < pos-1 ; i++) {
  p = p->next;
 }

t->next = p->next;		/* t next pointer points to the right node */
p->next = t; 			/* p->next points to t, so t is between left and right */


\end{minted}
\item Combine
\label{sec:org4bf0b5e}
\begin{minted}[breaklines=true,breakanywhere=true,frame=single]{c}
void Insert (int pos, int x){
  Node *t, *p;
  if (pos == 0) {
    t = new Node;
    t->data = x;
    t->next = first;
    first = t;
  }
  else if (pos > 0) {
    p = first;
    for (i = 0; i < pos-1 && p != NULL ; i++) {
      p = p->next;
    }

    if (p != NULL) {
      t = new Node;
      t->data = x;
      t->next = p->next;
      p->next = t;
    }
  }
}
\end{minted}
\item Special case: Insert at last only
\label{sec:orgc09ddd1}
\begin{minted}[breaklines=true,breakanywhere=true,frame=single]{c}
void InsertLast(int x){
  Node *t = new Node;
  t->data = x;
  t->next = NULL;

  /* no node in list */
  if (first == NULL) {
    first = last = t;
  }
  else {
    last->next = t;
    last = t;
  }
}
\end{minted}
\item Insert in a \textbf{SORTED} linked list, at a \textbf{SORTED} position
\label{sec:org91552df}
\begin{minted}[breaklines=true,breakanywhere=true,frame=single]{c}
/* Time: min O(1) max O(n) */
p = first;
q = NULL;

while (p != NULL && p->data < x) {
  q = p;
  p = p->next;
 }

t = new Node;
t->data = x;
t->next = q->next;
q->next = t;
\end{minted}
\end{enumerate}
\subsubsection{Deleting}
\label{sec:org2acb104}
\begin{enumerate}
\item Delete first node
\label{sec:org915a5e4}
\begin{minted}[breaklines=true,breakanywhere=true,frame=single]{c}
/* Time O(1)  */
Node *p = first;		/* arbitray pointer p poins to first */
first=first->next;		/* move first to point to next node */
delete p; 			/* delete the original first */

\end{minted}
\item Delete at given position
\label{sec:org07c3121}
\begin{minted}[breaklines=true,breakanywhere=true,frame=single]{c}
/* Time min O(1) max O(n) */
Node *p = first;
Node *q = NULL;

for (i = 0; i < pos-1 ; i++) {
  q = p;
  p = p->next;
 }

q->next = p->next;
delete p;

\end{minted}
\end{enumerate}
\subsubsection{Check if linkedlist is sorted}
\label{sec:orgc144797}
\begin{minted}[breaklines=true,breakanywhere=true,frame=single]{c}
/* Time O(n) max O(1) min */
int x = -32768;
Node *p = first;

while (p != NULL) {
  if (p->data < x) {
    return false;
  }
  x = p->data;
  p = p->next;
 }
return true;
\end{minted}
\subsubsection{Remove duplicate}
\label{sec:org34ee537}
\begin{minted}[breaklines=true,breakanywhere=true,frame=single]{c}
Node *p = first;
Node *q = first->next;
while (q != NULL) {
  if (p->data != q->data) {
    p = q;
    q = q->next;
  }
  else {
    p->next = q->next;
    delete q;
    q = p->next;
  }
 }
\end{minted}

\subsubsection{Reverse a linkedlist}
\label{sec:org9177f9d}
\begin{enumerate}
\item Interchange elements
\label{sec:org015edd2}
\begin{minted}[breaklines=true,breakanywhere=true,frame=single]{c}
p = first;
i = 0;
/* Copy to extra array */
while (p != NULL) {
  A[i] = p->data;
  p = p->next;
  i++;
 }
p = first;
i--;

/* Reverse copy back to list */
while (p != NULL) {
  p->data = A[i];
  i--;
  p = p->next;
 }
\end{minted}
\item Reversing links
\label{sec:orgb1cc34a}
\begin{minted}[breaklines=true,breakanywhere=true,frame=single]{c}
/* Setup 3 sliding pointers */
p = first;
q = NULL;
r = NULL;

while (p != NULL) {
  r = q;
  q = p;
  p = p->next;

  q->next = r;
 }

/* Update first */
first = q;
\end{minted}
\item Recursion
\label{sec:orgf558d5a}
\begin{minted}[breaklines=true,breakanywhere=true,frame=single]{c}
void Reverse(Node *q, Node *p){
  if (p != NULL) {
    Reverse(p, p->next);
    p->next = q;
  }
  else {
    first = q;

  }
}
\end{minted}
\end{enumerate}

\subsubsection{Joining/Append 2 linked list}
\label{sec:orgd1583a2}
\begin{minted}[breaklines=true,breakanywhere=true,frame=single]{c}
p = first;

/* traverse till the last node and stop */
while (p->next != NULL) {
  p = p->next;
 }

p->next = second; 		/* point last node to first node of the other list */
second = NULL; 			/* delete/free/NULL the extra pointer */
\end{minted}
\subsubsection{Merging 2 linkedlist}
\label{sec:orgce1c559}
\begin{minted}[breaklines=true,breakanywhere=true,frame=single]{c}
/* Create 2 pointers for the merged list */
Node *third, *last;

/* First loop */
if (first->data < second->data) {
  third = last = first;
  first = first->next;
  last->next = NULL;
 }
 else {
   third=last=second;
   second = second->next;
   last->next = NULL;
 }

while (first != NULL && second != NULL) {
  if (first->data < second->data) {
    last->next = first;
    last = first;
    first = first->next;
    last->next = NULL;
  }
  else {
    last->next = second;
    last = second;
    second = second->next;
    last->next = NULL;
  }
 }

if (first != NULL) {
  last->next = first;
 }
 else {
   last->next = second;
 }
\end{minted}
\subsubsection{Check for LOOP in}
\label{sec:orgf0ea99b}
\begin{enumerate}
\item LOOP: Last node points to some other nodes
\label{sec:orgb11365e}
\item LINEAR: Last node points to NULL
\label{sec:orgbc3f19d}
\begin{enumerate}
\item 
\label{sec:org2c7f655}
\begin{minted}[breaklines=true,breakanywhere=true,frame=single]{c}
int isLoop(Node *first){
  Node *p, *q;
  p = q = first;
  do
    {
      p = p-next;
      q = q->next;
      if (q != NULL) {
	q = q->next;
      }
      else {
	q = NULL;
      }
    } while (p != NULL && q != NULL);


  if (p == q) {
    return true;
  }
  else {
    return false;
  }


}
\end{minted}
\end{enumerate}
\end{enumerate}

\subsubsection{Circular linkedlist}
\label{sec:org33fd30e}
\begin{enumerate}
\item Last node points to first node
\label{sec:org8155811}
\item or a collection of nodes that are circularly connected
\label{sec:org9f1922e}
\item Use HEAD instead of FIRST
\label{sec:org88c449d}
\item Two representations:
\label{sec:org469194a}
\begin{enumerate}
\item HEAD/1st node -> 2nd node -> 3rd node -> 4th node -> back to HEAD/1st node
\label{sec:org261123c}
\item HEAD -> 1st -> 2nd -> 3rd -> 4th -> back to 1st
\label{sec:org78a90f3}
\end{enumerate}
\item How to display
\label{sec:org457afa6}
\begin{enumerate}
\item Loop display
\label{sec:org27c8afa}
\begin{minted}[breaklines=true,breakanywhere=true,frame=single]{c}
void Display(Node *p){
  do
    {
      printf("%d\n", p->data);
      p = p->next;
    } while (p != Head);

}
\end{minted}
\item Recursive display
\label{sec:org51b426b}
\begin{minted}[breaklines=true,breakanywhere=true,frame=single]{c}
void Display(Node *p){
  static int flag = 0;		/* so only 1 creation of int flag */
  if (p != Head || flag = 0) {
    flag = 1;
    printf("%d\n", p->data);
    Display(p->next);
  }
  flag = 0;
}
\end{minted}
\end{enumerate}
\item How to insert
\label{sec:orgd6b1efb}
\begin{enumerate}
\item After Head
\label{sec:org156eb32}
\begin{minted}[breaklines=true,breakanywhere=true,frame=single]{c}
Node *t;
Node *p = Head;

for (i = 0; i < pos-1 ; i++) {
  p = p->next;
 }

t = new Node;
t->data = x;
t->next = p->next;
p->next = t;
\end{minted}
\item Before Head
\label{sec:orgcf9bcda}
\begin{minted}[breaklines=true,breakanywhere=true,frame=single]{c}
Node *t = new Node;
t->data = x;
t->next = Head;

Node *p = Head;
while (p->next != Head) {
  p = p->next;
 }

p->next = t;
Head = t;
\end{minted}
\end{enumerate}
\item Delete
\label{sec:org8ad2eb4}
\begin{enumerate}
\item Delete from Given Position
\label{sec:org41b2aa5}
\begin{minted}[breaklines=true,breakanywhere=true,frame=single]{c}
p = Head;
for (i = 0; i < pos-2 ; i++) {
  p = p->next;
 }

q = p->next;
p->next = q->next;
x = q->data;
delete q;

\end{minted}
\item Delete Head
\label{sec:org5b22c6c}
\begin{minted}[breaklines=true,breakanywhere=true,frame=single]{c}
p = Head;
while (p->next != Head) {
  p = p->next;
 }

p->next = Head->next;
x = Head->data;
delete Head;
Head = p->next;
\end{minted}
\end{enumerate}
\end{enumerate}

\subsection{Doubly Linked List}
\label{sec:orga847e2d}
\subsubsection{a node has pointer to \textbf{NEXT} node and \textbf{PREVIOUS} node}
\label{sec:org89f5d90}
\subsubsection{Structure in C:}
\label{sec:org9ff4653}
\begin{minted}[breaklines=true,breakanywhere=true,frame=single]{c}
struct Node
{
  struct Node *prev;
  int data;
  struct Node *next;
};
\end{minted}
\subsubsection{Insertion}
\label{sec:org4794d8f}
\begin{enumerate}
\item Before first
\label{sec:orgcf2327e}
\begin{minted}[breaklines=true,breakanywhere=true,frame=single]{c}
Node *t = new Node;
t->data = x;

/* Modify links */
t->prev = NULL;
t->next = first;
first->prev = t;

/* Rename new first */
first = t;
\end{minted}
\item After given index min O(1) max O(n)
\label{sec:org6fd7cdc}
\begin{minted}[breaklines=true,breakanywhere=true,frame=single]{c}
Node *t = new Node;
t->data = x;

/* To reach the node before the insertion */
for (i = 0; i < pos-1 ; i++) {
  p = p->next
    }

/* Modify links */
t->next = p->next;		/* inserted node should point to the node on the RIGHT (p->next) */
t->prev = p;			/* inserted node should point to the node on the LEFT (p) */


/* for node on the RIGHT, its prev must point to the inserted node */
/* Must check if next is available, in case we insert at last then next is NULL */
if (p->next ! =NULL) {
  p->next->prev = t;
 }

p->next = t; 			/* node on the LEFT should point to the inserted node */

\end{minted}
\end{enumerate}
\subsubsection{Delete}
\label{sec:orgd4c97a1}
\begin{enumerate}
\item Delete 1st node
\label{sec:org6b46761}
\begin{minted}[breaklines=true,breakanywhere=true,frame=single]{c}
p = first;
first = first->next;
x = p->data;
delete p;

if (first != NULL) {
  first->prev = NULL;
 }
\end{minted}
\item Delete from given index
\label{sec:orgc0d7f7a}
\begin{minted}[breaklines=true,breakanywhere=true,frame=single]{c}
/* bring a pointer p upon given index */
p = first;

for (i = 0; i < pos-1 ; i++) {
  p = p->next;
 }

p->prev->next = p->next; 	/* LEFT node points to RIGHT node, skip CURRENT */

if (p->next != NULL) {		/* if RIGHT node exists */
  p->next->prev = p->prev;  	/* RIGHT node points to LEFT node, skip CURRENT */
 }

x = p->data;
delete p;
\end{minted}
\end{enumerate}
\subsubsection{Reverse}
\label{sec:orge979e23}
\begin{enumerate}
\item Display
\label{sec:orgecffd58}
\begin{minted}[breaklines=true,breakanywhere=true,frame=single]{c}
p = first;

while (p != NULL) {
  printf("%d\n", p->data);
  p = p->next

    }

\end{minted}
\item Reverse
\label{sec:org2994a66}
\begin{minted}[breaklines=true,breakanywhere=true,frame=single]{c}
p = first;

while (p != NULL) {
  temp = p->next;
  p->next = p->prev;
  p->prev = temp;
  p = p->prev;
 }

/* for last node, bring first there */
if (p!= NULL && p->next == NULL) {
  first = p;
 }
\end{minted}
\end{enumerate}
\subsubsection{Circular}
\label{sec:orga5fdc57}
\begin{enumerate}
\item Insert
\label{sec:org6ddbdab}
\item Display
\label{sec:org1a08351}
\end{enumerate}
\section{Comparing different linkedlists}
\label{sec:orgeadd123}
\subsection{Space: Doubly takes double the amount of pointers in Singly}
\label{sec:org790e75b}
\subsection{Insert: constant time for Linear Singly, Linear Doubly, Circular Doubly.  Nth time for Circular Singly}
\label{sec:orgdc31ba7}
\section{Array vs Linkedlist}
\label{sec:org4243f89}
\subsection{Creation: array in stack or heap, LL is always on heap}
\label{sec:org91664ed}
\subsection{Size: array fixed, LL can grow until heap is full}
\label{sec:org2735d81}
\subsection{LL takes extra space}
\label{sec:org12eaf87}
\subsection{Array access directly (faster), LL access sequentially (slower)}
\label{sec:org8648808}
\subsection{Data movement in array is more expensive}
\label{sec:org77f3087}
\section{Student challenge}
\label{sec:org9ce48c7}
\subsection{Find middle node of LL}
\label{sec:org1859bc3}
\subsubsection{1st solution}
\label{sec:org43d2c11}
\begin{enumerate}
\item find length of LL => say 7
\label{sec:org9c7e133}
\item reach middle node => 7/2 \textasciitilde{} 4
\label{sec:orga1ee2a1}
\item move the pointer 4-1 times
\label{sec:orga9cc77e}
\end{enumerate}
\subsubsection{2nd solution}
\label{sec:orgcd2f2fa}
\begin{enumerate}
\item use 2 pointers, 1 move 2 space, other move 1 space
\label{sec:orgd983d03}
\end{enumerate}
\subsubsection{3rd solution}
\label{sec:org5e3de2e}
\begin{enumerate}
\item push each node to a stack
\label{sec:org2aa645a}
\item pop out the node at stack number floor(stack size / 2)
\label{sec:org1e47d16}
\end{enumerate}
\subsection{Find intersection of 2 LL}
\label{sec:org4b0ac86}
\subsubsection{Eg: LL1->LL3 and LL2->LL3 => Both LL1 and LL2 has LL3 as common. For example:}
\label{sec:org2de5f6f}
\begin{enumerate}
\item LL1: 8->6->3->9->10->4->2->12
\label{sec:org8bc06fc}
\item LL2: 20->30->40->10->4->2->12
\label{sec:orgfc82715}
\end{enumerate}
\subsubsection{We need to find the starting common point, i.e. block \textbf{10}?}
\label{sec:orgd373bf9}
\begin{enumerate}
\item We traverse the 1st LL or 2nd LL till the end.
\label{sec:org6f3a0f7}
\item Then from end, we traverse back, if the previous block is different,
\label{sec:org3de246d}
\item then current block is \textbf{intersection point}
\label{sec:orge3c6257}
\item But cannot traverse back in Single LL?? Use Stack
\label{sec:orgb53a5a5}
\begin{enumerate}
\item Traverse each LL, store address in a stack
\label{sec:orgb6e3c44}
\item Then compare two stacks, whenever the address differ, that location is the \textbf{intersection}
\label{sec:org1af893e}
\end{enumerate}
\item Suppose our LL1 and LL2 has the following address
\label{sec:orgbc54f56}
\begin{enumerate}
\item LL1: length 8
\label{sec:orgba92eef}
\begin{center}
\begin{tabular}{rr}
Data & Address\\
\hline
8 & 100\\
6 & 110\\
3 & 130\\
9 & 150\\
\textbf{10} & \textbf{200}\\
4 & 220\\
2 & 240\\
12 & 260\\
\end{tabular}
\end{center}
\item LL2: length 7
\label{sec:org1a4a35c}
\begin{center}
\begin{tabular}{rr}
Data & Address\\
\hline
20 & 300\\
30 & 310\\
40 & 330\\
\textbf{10} & \textbf{200}\\
4 & 220\\
2 & 240\\
12 & 260\\
\end{tabular}
\end{center}
\item Record address in a stack. \textbf{Note}: first address goes to bottom:
\label{sec:org625d9af}
\begin{center}
\begin{tabular}{rr}
Stack 1 & Stack 2\\
\hline
260 & 260\\
240 & 240\\
220 & 220\\
\textbf{200} & \textbf{200}\\
150 & 330\\
130 & 310\\
110 & 300\\
100 & \\
\end{tabular}
\end{center}
\begin{enumerate}
\item Comparing the two stacks, pops out an address, if same, delete
\label{sec:org18327d5}
\item Keep track of the previous address that we pop out.
\label{sec:orge74f13c}
\begin{center}
\begin{tabular}{rr}
Stack 1 & Stack 2\\
\hline
260 & 260\\
240 & 240\\
220 & 220\\
200 & 200\\
\textbf{150} & \textbf{330}\\
130 & 310\\
110 & 300\\
100 & \\
\end{tabular}
\end{center}
\item Addresses \textbf{150} and \textbf{330} are different => The intersection is the node before
\label{sec:org859a04f}
\item i,e, node with address 200, which contains data \textbf{10}
\label{sec:orgae1a587}
\end{enumerate}
\end{enumerate}
\end{enumerate}
\subsection{Sparse matrix using LL}
\label{sec:org076b9bd}
\subsubsection{Suppose we have the following matrix 5 x 6}
\label{sec:org19e9da1}

\begin{center}
\begin{tabular}{llrrrrrr}
. & / & \textbf{0} & \textbf{1} & \textbf{2} & \textbf{3} & \textbf{4} & \textbf{5}\\
\hline
\textbf{0} & / & 0 & 0 & 0 & 0 & 8 & 0\\
\textbf{1} & / & 0 & 0 & 0 & 7 & 0 & 0\\
\textbf{2} & / & 5 & 0 & 0 & 0 & 9 & 0\\
\textbf{3} & / & 0 & 0 & 0 & 0 & 0 & 3\\
\textbf{4} & / & 6 & 0 & 0 & 4 & 0 & 0\\
\end{tabular}
\end{center}

\subsubsection{We want to \textbf{AVOID STORING ZEROES}}
\label{sec:org67a387f}
\subsubsection{Coordinate, for number 7 is at (1,3) 1st row, 3rd column}
\label{sec:org0894341}
\subsubsection{For each row, we represent the non zero as an LL. Eg:}
\label{sec:orgb5c42d9}
\begin{enumerate}
\item \textbf{Row 0}: |-4-|-8-|-NULL-|
\label{sec:org149fd4d}
\begin{enumerate}
\item Row 0 has element at 4th column, data is 8, it has no other, so pointer next is NULL
\label{sec:orgdbd4faf}
\end{enumerate}
\item \textbf{Row 1}: |-3-|-7-|-NULL-|
\label{sec:orge577ccc}
\item \textbf{Row 2}: |-0-|-5-|------| --> |-4-|-9-|-NULL-|
\label{sec:orgc56a758}
\begin{enumerate}
\item Here Row 2 at 0th column has element 5, there is also element 9 at 4th column, so \textbf{next} point to this.
\label{sec:org54205dc}
\end{enumerate}
\item \textbf{Row 3}: |-5-|-3-|-NULL-|
\label{sec:org2337b2e}
\item \textbf{Row 4}: |-0-|-6-|------| --> |-3-|-4-|-NULL-|
\label{sec:org16cf40b}
\end{enumerate}
\subsubsection{This is called an \textbf{ARRAY OF LINKED LIST}}
\label{sec:org16903be}
\subsubsection{So, the Node structure contains: \textbf{Column}, \textbf{Value}, \textbf{Next}}
\label{sec:org412f958}
\subsubsection{To make this structure in C:}
\label{sec:orgaccd75b}
\begin{minted}[breaklines=true,breakanywhere=true]{c}
struct Node
{
  int col;
  int val;
  struct Node *next;
};
\end{minted}
>
\subsubsection{We need the number of row, so we can create an array of size row}
\label{sec:orge25f273}
\subsubsection{In our example, we have 5 rows x 6 cols. Generalized to m x n matrix, where \textbf{m = 5}}
\label{sec:org837e5a4}
\begin{minted}[breaklines=true,breakanywhere=true]{c}
/* Create array of size m */
Node *A[m];

/* For each row, we create a new node */
A[0] = new Node;
\end{minted}
>
\subsection{Polynomial representations}
\label{sec:org7567c28}
\subsubsection{Consider: P(x) = 4x\textsuperscript{3} + 9x\textsuperscript{2} + 6x + 7}
\label{sec:org3248fdd}
\subsubsection{We will construct each term as a node that has: coefficient, exponent and next pointer}
\label{sec:orgab3ab9e}
\begin{minted}[breaklines=true,breakanywhere=true,frame=single]{c}
struct Node
{
  int coeff;
  int exp;
  struct Node *next;
};
\end{minted}

\section{Codes}
\label{sec:orga9faf18}
\subsection{Implement in C: \url{linkedlistBasic.c}}
\label{sec:org15c7456}
\subsection{Implement in C++: \url{linkedlistBasic.cpp}}
\label{sec:org430279c}
\subsection{Circular linkedlist in C: \url{circularLinkedList.c}}
\label{sec:orgc4f225e}
\subsection{Circular Doubly LinkedList in C: \url{circularDoubly.c}}
\label{sec:orga1bb137}
\subsection{Doubly linkedlist in C: \url{doubleLinkedList.c}}
\label{sec:org0a3748e}
\subsection{Student challenge intersection: \url{challengeIntersection.c}}
\label{sec:org170efd2}
\subsection{Student challenge middle node: \href{challengeMiddle.c}{findMiddle}}
\label{sec:orga82ad44}
\subsection{Student challenge sparse matrix: \href{challengeSparseMatrix.c}{sparseMatrix}}
\label{sec:org84c2f4a}
\subsection{Basic Stack implementation: \href{stackBasic.c}{stackBasic}}
\label{sec:orgee6b159}
\end{document}