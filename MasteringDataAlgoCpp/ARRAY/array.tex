% Created 2020-09-27 Sun 14:23
% Intended LaTeX compiler: pdflatex
\documentclass{article}
\usepackage[utf8]{inputenc}
\usepackage[T1]{fontenc}
\usepackage{graphicx}
\usepackage{grffile}
\usepackage{longtable}
\usepackage{wrapfig}
\usepackage{rotating}
\usepackage[normalem]{ulem}
\usepackage{amsmath}
\usepackage{textcomp}
\usepackage{amssymb}
\usepackage{capt-of}
\usepackage{hyperref}
\date{\today}
\title{}
\hypersetup{
 pdfauthor={},
 pdftitle={},
 pdfkeywords={},
 pdfsubject={},
 pdfcreator={Emacs 27.1 (Org mode 9.3)}, 
 pdflang={English}}
\begin{document}

\tableofcontents





\section{Intro to array}
\label{sec:org8a80096}
\subsection{array on integer}
\label{sec:org8172de6}
\subsection{Declaring arrays}
\label{sec:org21a09d5}
\subsubsection{int A[5];   where we get garbage values}
\label{sec:org90d8e5a}
\subsubsection{int A[5] = \{2,4,6,8,10\}; all values initialized}
\label{sec:org6c88856}
\subsubsection{int A[5] = \{2.4\}; only first 2 are initialized, rest is 0}
\label{sec:orgab3949b}
\subsubsection{int A[5] = \{0\}; all zero}
\label{sec:orgc9cd4fe}
\subsubsection{int A[] = \{2,4,6,8,10\}; automatically create A[5]}
\label{sec:orge7d6786}
\subsection{Traverse using for loop}
\label{sec:org9a06cad}
\subsection{To print an element at position 2}
\label{sec:org7e857bb}
\subsubsection{A[2];}
\label{sec:orgd8356c4}
\subsubsection{2[A];}
\label{sec:org878a616}
\subsubsection{*(A+2);}
\label{sec:org49bbe17}
\section{Static vs Dynamic array}
\label{sec:org8b58cba}
\subsection{Static}
\label{sec:orgb49acc7}
\subsubsection{size cannot be modified.  Memories created on STACK}
\label{sec:org5452ab0}
\subsubsection{C: size decided at compilation time}
\label{sec:org6c99163}
\subsubsection{C++: size at run time. Eg. cin >> n; int A[n];}
\label{sec:org2db8f53}
\subsection{Dynamic}
\label{sec:org0c7c02a}
\subsubsection{on HEAP}
\label{sec:org0124f25}
\begin{enumerate}
\item Create pointer int *p on STACK
\label{sec:org73d848c}
\item C++: p = new int[5]; create 5 integer array on HEAP
\label{sec:org416b901}
\item C: p = (int*)malloc(sizeof(int)*5);
\label{sec:org5a31ef0}
\end{enumerate}
\subsubsection{Note: remember to free memory}
\label{sec:orgf444346}
\begin{enumerate}
\item C++: delete []p; if p is used for an array we use []
\label{sec:orgc84efc5}
\item C: free(p)
\label{sec:org8d3fc56}
\end{enumerate}
\subsubsection{Access on heap;}
\label{sec:org92aaeaa}
\begin{enumerate}
\item p[0] = 5;
\label{sec:org1bf3c64}
\end{enumerate}
\section{Demo static dynamic array}
\label{sec:org0690ef9}
\begin{verbatim}
#include <stdio.h>
#include <stdlib.h>

int main(){
  int A[5] = {2,4,6,8,10};
  int *p;
  int i;

  p=(int*)malloc(sizeof(int) * 5);
  p[0] = 3;
  p[1] = 5;
  p[2] = 7;
  p[3] = 9;
  p[4] = 11;

  for (int i = 0; i < 5 ; i++) {
    printf("%d\t%d\n", A[i], p[i]);
  }


  return 0;
}

\end{verbatim}
\section{Increase array size}
\label{sec:org67470d9}
\subsection{int *p = new int[5]}
\label{sec:org9902107}
\subsection{Take another pointer: int *q = new int[10] => Create larger array separately}
\label{sec:org59262aa}
\subsection{Copy p[i] onto q[i]}
\label{sec:org413c228}
\subsection{delete/free memory in p}
\label{sec:org414c5f8}
\subsection{tells p to to point to q => both p and q points to the same larger array}
\label{sec:orgaa99032}
\subsection{free q}
\label{sec:org4dbccab}
\subsection{demo}
\label{sec:org9976efd}
\begin{verbatim}
#include <stdio.h>
#include <stdlib.h>

int main(){
  int *p, *q;

  p = (int*)malloc(sizeof(int) * 5);
  p[0] = 3;
  p[1] = 5;
  p[2] = 7;
  p[3] = 9;
  p[4] = 11;


  /* for (int i = 0; i < 5 ; i++) { */
  /* 	 printf("%d\n", p[i]); */
  /* } */

  q = (int*)malloc(sizeof(int) * 10);

  for (int i = 0; i < 5 ; i++) {
    q[i] = p[i];
  }

  free(p);
  p = q;
  q = NULL;

  for (int i = 0; i < 5 ; i++) {
    printf("%d\n", p[i]);
  }

  return 0;
}

\end{verbatim}
\section{2D array}
\label{sec:org21454ca}
\subsection{Method 1: int A[3][4] => 3 row, 4 col on STACK}
\label{sec:orga17bfc5}
\subsubsection{Memory allocates like a 1D array of 12 memory blocks}
\label{sec:orgaba5ad2}
\subsection{Method 2: int *A[3] => array of int pointers of size 3 on STACK, actual array on HEAP}
\label{sec:org193f43b}
\subsubsection{block 0 [ ] -> want array of size 4 here | | | | |}
\label{sec:orga5e0cd8}
\subsubsection{block 1 [ ] -> want array of size 4 here | | | | |}
\label{sec:orgb2a7563}
\subsubsection{block 2 [ ] -> want array of size 4 here | | | | |}
\label{sec:org5ece4ff}
\subsubsection{A[0] = new int[4] => create array of size 4 for block 0}
\label{sec:org61b2d4c}
\subsubsection{A[1] = new int[4] and A[2] = new int[4]}
\label{sec:org1575f66}
\subsection{Method 3: int **A; everything on HEAP}
\label{sec:orgc839100}
\subsection{A = new int*[3] create array of int pointers (like above) on HEAP}
\label{sec:org014a23d}
\subsection{A[0] = new int[4] on HEAP}
\label{sec:org2bac38c}
\subsection{A[1] = new int[4] on HEAP}
\label{sec:orgf718875}
\subsection{A[2] = new int[4] on HEAP}
\label{sec:orgc05978a}
\subsection{Demo : `\url{2darray.c}`}
\label{sec:org841103b}
\section{1D Array in compilers}
\label{sec:org8441a62}
\subsection{int x = 10; compiler allocates address for x and store 10 at that address}
\label{sec:org9fc6a14}
\subsection{Compiler memory to address}
\label{sec:orgfd78ed0}
\subsection{int A[5] = \{\ldots{}..\};}
\label{sec:org4e59aa2}
\subsection{A[i] = Base index + index *  sizeof (data type)}
\label{sec:orga6ba84e}
\subsection{A[3] = L0 + 3 * 2}
\label{sec:orgeee39cb}
\subsection{If index starts at 1:  A[i] = Base index + (index-1)*sizeof(data type)}
\label{sec:org5a8efc5}
\section{2D Array in compilers}
\label{sec:org4880e0d}
\subsection{ROW MAJOR MAPPING}
\label{sec:orgea97928}
\subsubsection{Elements store row by row in A[m x n]}
\label{sec:orgfb52e6b}
\subsubsection{A = a00 a01 a02 a03 | a10 a11 a12 a13 | a20 a21 a22 a23 |}
\label{sec:org9b5e58e}
\subsubsection{Say we access A[1][2] and say a00 has address 200}
\label{sec:org485de3e}
\begin{enumerate}
\item A[1][2] = 200 + [4 + 2]*sizeof(int)
\label{sec:org5775452}
\end{enumerate}
\subsubsection{In general A[i][j] = \textbf{L0 + [i*n+j]*sizeof(data type)}}
\label{sec:org4e06554}
\subsubsection{If index starts at 1:  A[i][j] = L0 + [(i-1)*n+(j-1)]*sizeof(data type)}
\label{sec:org0433bfd}
\subsection{COL MAJOR MAPPING}
\label{sec:org6c4eb09}
\subsubsection{Map colum by colum}
\label{sec:org450e3ed}
\subsubsection{A = a00 a10 a20 | a01 a11 a21 | a02 a12 a21 | a03 a13 a23 |}
\label{sec:org7514853}
\subsubsection{Say we want A[1][2]}
\label{sec:orgb6baf00}
\begin{enumerate}
\item A[1][2] = 200 + [2 * 3 + 1]*sizeof(int)
\label{sec:orgd672be3}
\end{enumerate}
\subsubsection{In general, \textbf{A[i][j] = L0 + [j*m + i]*sizeof)(data type)}}
\label{sec:org3ffc12b}
\section{4D Array}
\label{sec:org2a9a788}
\subsection{Type A[d1][d2][d3][d4]}
\label{sec:orgd454744}
\subsection{Row major Add(A[i][i2][i3][i4]) = L0 + [i1*d2*d3*d4 + i2*d3*d4 + i3*d4 +i4]*sizeof(data)}
\label{sec:org3e3d4fc}
\subsection{Col major Add(A[i1][i2][i3][i4]) = L0 + [i4*d1*d2*d3 + i3*d1*d2 + i2*d1 + i1]*sizeof(data)}
\label{sec:org3a362d6}
\section{For nD array}
\label{sec:org3fac583}
\subsection{Row major mapping: L0 + SUM\textsubscript{p} from 1 to n [  (i\textsubscript{p}) * product\textsubscript{q} = p + 1 to n of dq] * sizeof(datetype)}
\label{sec:org2d7cd3d}
\subsubsection{O(n\textsuperscript{2})}
\label{sec:org06ca89d}
\subsubsection{If rewrite by taking commons => O(n) --> \textbf{HOMER'S RULE}}
\label{sec:orgd2f0992}
\section{3D Array}
\label{sec:org216e432}
\subsection{int A[l][m][n]}
\label{sec:org463fa20}
\subsection{Row major Addr(A[i][j][k]) = L0 + [i*m*n + j*n + k] + sizeof(datatype)}
\label{sec:orgbbcd360}
\subsection{Colum major Addr(A[i][j][k]) = L0 + [k*m*l + j*l + i] + sizeof(datatype)}
\label{sec:orgeb6225a}

\section{Quiz}
\label{sec:orgb11c417}

\subsection{1. A[1\ldots{}.10][1\ldots{}15] = A[m][n]}
\label{sec:orgc7e35eb}

\subsubsection{L0 = 100}
\label{sec:orgdaa38c8}

\subsubsection{Row major Addr(A[i][j]) = L0 + [(i-1)*n+(j-1)]*sizeof(data type)}
\label{sec:org3fbc03d}

\subsubsection{100 + [(i-1)*15+(j-1)]*1}
\label{sec:org5743141}

\subsubsection{100 + (15i- 15 + j - 1)*1}
\label{sec:org58fee1c}

\subsubsection{100 + 15i - 15 + j - 1}
\label{sec:org5f72e13}

\subsubsection{84 + 15 i + j}
\label{sec:orgf1dbae3}

\subsection{2. unsigned int x[4][3] = \{\ldots{}\ldots{}\}. Printf("\%u, \%u, \%u", x + 3, *(x+3), *(x+2)+3)}
\label{sec:org0f7b496}

\subsubsection{1 2 3}
\label{sec:org7391f94}

\subsubsection{\textbf{4} 5 6}
\label{sec:org5d0219d}

\subsubsection{7 8 9}
\label{sec:org2998a01}

\subsubsection{10 11 12}
\label{sec:orgd1fc024}

\subsubsection{A = a00 a01 a02 | a03 a04 a05 | a06 a07 a08 | a09 a10 a11 |}
\label{sec:orgd26a38c}

\subsubsection{\%u, x + 3 => 2000 + (3*int) = > \textbf{2012 address}}
\label{sec:orgb3ce56b}

\subsubsection{\%u, *(x+3) => Gets value of address 2012  =  4}
\label{sec:org1267dcd}

\subsubsection{\%u, \textbf{(x+2) + 3 => *6}}
\label{sec:org7bf7728}
\subsubsection{testing: `\url{quizarray2.c}`}
\label{sec:org9917b76}

\subsection{4. ?X[?][?][?]}
\label{sec:org61f92d4}

\subsubsection{t0 = i*1024}
\label{sec:org321cd87}

\subsubsection{t1 = j + 32}
\label{sec:org6c641b6}





\section{Array ADT}
\label{sec:orgcb40b3f}
\subsection{Perform various operations on an array}
\label{sec:org0cacd38}
\subsection{Data}
\label{sec:org399deda}
\subsubsection{Array space: say 10}
\label{sec:orga8840b1}
\subsubsection{Size}
\label{sec:org397dcf7}
\begin{enumerate}
\item static
\label{sec:orgbf0a076}
\begin{enumerate}
\item int A[10]
\label{sec:org461df5f}
\end{enumerate}
\item dynamic
\label{sec:org3c3d68a}
\begin{enumerate}
\item int * A
\label{sec:org0f9fd8a}
\item A = new int[size]
\label{sec:org7bc2d6d}
\end{enumerate}
\end{enumerate}
\subsubsection{length}
\label{sec:orgc663e88}
\subsection{Operations}
\label{sec:orgf87725b}
\subsubsection{display: printf ("\%d", A[i]) in for loop}
\label{sec:org56b074a}
\subsubsection{add/append}
\label{sec:org4e035e8}
\begin{enumerate}
\item Add new element at \textbf{END} of the array
\label{sec:org9982f58}
\item A[Length] = x; length++;
\label{sec:org7ec26ea}
\end{enumerate}
\subsubsection{insert}
\label{sec:org4082ff3}
\begin{enumerate}
\item shifted forward to allow space
\label{sec:orgeb28a16}
\item start from last, copy prev last and \textbf{STOP} until reach insertion point
\label{sec:org4c2182c}
\item pseudocode
\label{sec:orge12f156}
\begin{verbatim}
for (i = length; i > index ; i--) {
  A[i] = A[i-1];
}

A[index] = x;
length++;

\end{verbatim}
\end{enumerate}
\subsubsection{delete}
\label{sec:orgfc818e0}
\begin{enumerate}
\item delete(index)
\label{sec:org78588db}
\item x = A[index]
\label{sec:org0fdbde4}
\item shift to occupy blank space
\label{sec:org6232956}
\item pseudocode
\label{sec:orgfeec43f}
\begin{verbatim}
for (i = index; i < Length-1 ; i++) {
  A[i] = A[i+1];
 }
Length--;
\end{verbatim}
\item Min time: 2 constant, Max time: n+2
\label{sec:org2a7b65f}
\end{enumerate}
\subsubsection{Linear search}
\label{sec:org80ee9c2}
\begin{enumerate}
\item assume unique
\label{sec:orgf7dcd67}
\item Use a key
\label{sec:orgfd238b0}
\end{enumerate}
\subsection{Demo: `\url{arrayADT.c}`}
\label{sec:org3eb825e}
\end{document}